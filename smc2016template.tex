% -----------------------------------------------
% Template for SMC 2016
% adapted from the template for SMC 2012 and 2011, which were adapted from that of SMC 2010
% -----------------------------------------------

\documentclass{article}
\usepackage{smc2016}
\usepackage{times}
\usepackage{ifpdf}
\usepackage[english]{babel}
\usepackage{cite}

%%%%%%%%%%%%%%%%%%%%%%%% Some useful packages %%%%%%%%%%%%%%%%%%%%%%%%%%%%%%%
%%%%%%%%%%%%%%%%%%%%%%%% See related documentation %%%%%%%%%%%%%%%%%%%%%%%%%%
%\usepackage{amsmath} % popular packages from Am. Math. Soc. Please use the 
%\usepackage{amssymb} % related math environments (split, subequation, cases,
%\usepackage{amsfonts}% multline, etc.)
%\usepackage{bm}      % Bold Math package, defines the command \bf{}
%\usepackage{paralist}% extended list environments
%%subfig.sty is the modern replacement for subfigure.sty. However, subfig.sty 
%%requires and automatically loads caption.sty which overrides class handling 
%%of captions. To prevent this problem, preload caption.sty with caption=false 
%\usepackage[caption=false]{caption}
%\usepackage[font=footnotesize]{subfig}


%user defined variables
\def\papertitle{Rethinking the audio workstation : tree-based sequencing with i-score and the LibAudioStream}
\def\firstauthor{Jean-Michaël Celerier}
\def\secondauthor{Myriam Desainte-Catherine}
\def\thirdauthor{Stéphane Letz}

% adds the automatic
% Saves a lot of ouptut space in PDF... after conversion with the distiller
% Delete if you cannot get PS fonts working on your system.

% pdf-tex settings: detect automatically if run by latex or pdflatex
\newif\ifpdf
\ifx\pdfoutput\relax
\else
   \ifcase\pdfoutput
      \pdffalse
   \else
      \pdftrue
\fi

\ifpdf % compiling with pdflatex
  \usepackage[pdftex,
    pdftitle={\papertitle},
    pdfauthor={\firstauthor, \secondauthor, \thirdauthor},
    bookmarksnumbered, % use section numbers with bookmarks
    pdfstartview=XYZ % start with zoom=100% instead of full screen; 
                     % especially useful if working with a big screen :-)
   ]{hyperref}
  %\pdfcompresslevel=9

  \usepackage[pdftex]{graphicx}
  % declare the path(s) where your graphic files are and their extensions so 
  %you won't have to specify these with every instance of \includegraphics
  \graphicspath{{./figures/}}
  \DeclareGraphicsExtensions{.pdf,.jpeg,.png}

  \usepackage[figure,table]{hypcap}

\else % compiling with latex
  \usepackage[dvips,
    bookmarksnumbered, % use section numbers with bookmarks
    pdfstartview=XYZ % start with zoom=100% instead of full screen
  ]{hyperref}  % hyperrefs are active in the pdf file after conversion

  \usepackage[dvips]{epsfig,graphicx}
  % declare the path(s) where your graphic files are and their extensions so 
  %you won't have to specify these with every instance of \includegraphics
  \graphicspath{{./figures/}}
  \DeclareGraphicsExtensions{.eps}

  \usepackage[figure,table]{hypcap}
\fi

%setup the hyperref package - make the links black without a surrounding frame
\hypersetup{
    colorlinks,%
    citecolor=black,%
    filecolor=black,%
    linkcolor=black,%
    urlcolor=black
}


% Title.
% ------
\title{\papertitle}

% Authors
% Please note that submissions are NOT anonymous, therefore 
% authors' names have to be VISIBLE in your manuscript. 
%
% Single address
% To use with only one author or several with the same address
% ---------------
%\oneauthor
%   {\firstauthor} {Affiliation1 \\ %
%     {\tt \href{mailto:author1@smcnetwork.org}{author1@smcnetwork.org}}}

%Two addresses
%--------------
% \twoauthors
%   {\firstauthor} {Affiliation1 \\ %
%     {\tt \href{mailto:author1@smcnetwork.org}{author1@smcnetwork.org}}}
%   {\secondauthor} {Affiliation2 \\ %
%     {\tt \href{mailto:author2@smcnetwork.org}{author2@smcnetwork.org}}}

% Three addresses
% --------------
 \threeauthors
   {\firstauthor} {Affiliation1 \\ %
     {\tt \href{mailto:author1@smcnetwork.org}{author1@smcnetwork.org}}}
   {\secondauthor} {Affiliation2 \\ %
     {\tt \href{mailto:author2@smcnetwork.org}{author2@smcnetwork.org}}}
   {\thirdauthor} { Affiliation3 \\ %
     {\tt \href{mailto:author3@smcnetwork.org}{author3@smcnetwork.org}}}


% ***************************************** the document starts here ***************
\begin{document}
%
\capstartfalse
\maketitle
\capstarttrue
%
\begin{abstract}
Place your abstract at the top left column on the first page.
Please write about 150--200 words that specifically highlight the purpose of your work,
its context, and provide a brief synopsis of your results.
Avoid equations in this part.
\end{abstract}
\section{Introduction}
\section{Existing works}
- Audiographe, etc.

\section{Context}
\subsection{Description i-score}
- interactivité
-> mapping and JS

\subsection{Description LibAudioStream}

\section{Audio novelties}
-> donner sémantique de flux des Stream Group, Send, Return.

- Comme la durée de chaque contrainte peut varier avec le ralentissement, on utilise principalement 
des dates symboliques
-> Processus audio dans i-score : 
- FX => Supports FaUST.
- Instrument.
- Send.
- Return.
- Mixing.
- Hiérarchie à profondeur arbitraire.

Faire graphe pour une Time Constraint et donner un exemple avec effets appliqués sur scénario.
Expliquer graphe hiérarchique de dépendances : penser au cas ou un a un return dans une hiérarchie puis un send à un niveau supérieur; il faut faire le grpahe de A à Z et s'assurer qu'il ne soit pas cyclique

1er cas : 
Un son avec une piste d'effets.

2eme cas : scénario hiérarchique, boucle

Cas de la boucle avec un coup A, un coup B selon la condition ? 
-> exécution d'un timenode doit reset le flux.

Piste send / return : permet de maintenir les queues de reverb.

\section{Routing, multi-channels, etc.}
-> mettre maquettes track mix
\section{UI}

\section{Conclusion}
-> lackluster areas : 
- MIDI support (but OSC)
- no musical time information : first aimed for artists, 
but improvements could be the waiting of triggering on some measure of time.
- "play from anywhere"
- audio input ?
- correction de latence ?
\begin{acknowledgments}
    ANRT, Blue Yeti, Magali
\end{acknowledgments} 

%%%%%%%%%%%%%%%%%%%%%%%%%%%%%%%%%%%%%%%%%%%%%%%%%%%%%%%%%%%%%%%%%%%%%%%%%%%%%
%bibliography here
\bibliography{smc2016template}

\end{document}
